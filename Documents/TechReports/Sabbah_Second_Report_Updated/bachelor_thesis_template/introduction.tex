\chapter{Introduction}
\label{chap:intro}

A microcontroller is a small computer on a single integrated circuit containing a processor core, memory, and programmable input/output peripherals\cite{Wikidef:URL}. Arduino Uno is based on the \href{http://www.atmel.com/devices/atmega328.aspx?tab=overview}{ATmega328} microcontroller. It is a popular and easy to work with microcontroller board. We provide a simulation for Arduino Uno, ATmega328 microcontroller and the experience of working with them.

\section*{Motivation} \label{sec:s1}
In educational context, microcontroller boards are considered a practical way for learning and practising embedded systems programming. These boards provide the user with the opportunity to learn to program and implement hardware circuits. Cost is a problem that stands against the growing use of microcontrollers mainly in educational context. This problem results in the unavailability of these boards in many universities, thus a lot of students do not have access to them. Making a simulator for these board would be a solution for this problem. Several simulators for Arduino have been implemented, but none of them can replace the the real life experience of the actual board and hardware components.

\section*{Aim of the project}
The aim of this project is to provide a real time hardware simulator for a commonly used microcontroller board (Arduino Uno). This simulator is as close as possible to real experience where it covers all aspects of an embedded system. It gives the user the ability to work with hardware components, hardware interfacing, board setting, code compilation and uploading. The user will be able to write, compile and upload Arduino code using the Arduino IDE. He will be able to implement the circuitry and hardware components virtually and connect them to Arduino Uno virtual board. He can choose from a scalable library of the most common hardware components. The output of the code and hardware connections is reflected on the board and circuitry. By providing these aspects, users would not need to buy the actual board or any hardware components.
