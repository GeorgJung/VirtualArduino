\documentclass[a4paper]{article}

\usepackage[T1]{fontenc}
\usepackage[adobe-utopia]{mathdesign}
\usepackage[protrusion=true,expansion=true]{microtype}
\usepackage{xcolor}

\definecolor{darkblue}{rgb}{0,0,0.5}

\usepackage[colorlinks=true,
        urlcolor=darkblue,
        anchorcolor=darkblue,
        linkcolor=darkblue,
        citecolor=darkblue,
        pdfauthor={Ebtessam Zoheir, Ahmed Sabbah},
        pdfkeywords={},
        pdftitle={Virtual Arduino},
        pdfsubject={Bachelor topic proposal for the faculty for MET,
          the German University in Cairo GUC
          (http://met.guc.edu.eg/)}]{hyperref}
\usepackage{url}

\author{Ebtessam Zoheir, Ahmed Sabbah}
\title{Virtual Arduino. A project proposal}

\begin{document}

\maketitle

\begin{abstract}
  
\section{Introduction}
\par The use of microcontroller evaluation boards is becoming more popular year by year. They are widely used for educational purposes in most universities. It’s considered the most efficient way to learn and practice embedded system programming.

\section{Motivation}
Problem is that not all students have access to these boards, the reason can be money or it’s unavailability in their country. Several simulators are already being used by some students but these simulators can never replace the real life experience of the actual board and actual hardware components.

\section{Project}
Our proposal is to make a real time hardware simulator that is as close as possible to the real experience, the simulator will cover all aspects of an embedded system including hardware components, hardware interfacing, hardware failure, board setting and code compiling and uploading. Another advantage of having the simulation done virtually is that it gives the user a wider variety of hardware components to work with that may not be available or may be too expensive. The simulator library will be very flexible and easy to expand to include future components and newer versions of the boards.\\

\noindent \underline{List if tasks to be done:}

\begin{itemize}
\item Compiler, IDE and board software.
\item Hardware components library.
\item Circuitry simulation.
\item Hardware failures.
\item Mini tutorials.
\item GUI for all previous features.
\end{itemize}
\end{abstract}

\end{document}

%%% Local Variables: 
%%% mode: latex
%%% TeX-master: t
%%% End: 
