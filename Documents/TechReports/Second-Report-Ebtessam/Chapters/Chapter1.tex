\lhead{\emph{Introduction}}  % Set the left side page header to "Introduction"

\chapter{Introduction}

The use of microcontroller evaluation boards is becoming more popular year by year. The post by Oskay, W. on \href{http://electronics.stackexchange.com/questions/2324/why-are-atmel-avrs-so-popular}{Why are Atmel AVRs so popular?} on April 22nd 2010 gives insight on the importance of Arduino and similar AVR based boards. The fact that the Arduino is easily programmed in C, has built-in Analog Digital Converter (ADC), Pulse Width Modulation (PWM) and has good cross-platform support makes it a very efficient way to learn and practice embedded system programming.

Given the importance of AVR microcontroller boards, acknowledging the fact that not all students have access to such evaluation boards we have decided that a simulator which could virtually replace the Arduino experience would be useful. Already existing simulators do not provide close to real life experience and hence cannot replace the actual Arudino. Differences between the Virtual Arduino project and already existing ones will be discussed later in the paper.

Our aim is to make a real time hardware simulator that is as close as possible to the real life experience. The simulator will later achieve this by covering all aspects of an embedded system including hardware components, hardware interfacing, hardware failure, board setting, code compiling and uploading and finally visualizing the whole process. Another advantage is that it will give the user a wider variety of hardware components to work with that may not be available or may be too expensive. The simulator library will be very flexible and easily scalable to include future components and newer versions of the boards.

\clearpage  % Introduction ended, start a new page
